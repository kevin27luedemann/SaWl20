% Für Bindekorrektur als optionales Argument "BCORfaktormitmaßeinheit", dann
% sieht auch Option "twoside" vernünftig aus
% Näheres zu "scrartcl" bzw. "scrreprt" und "scrbook" siehe KOMA-Skript Doku
\documentclass[12pt,a4paper,titlepage,headinclude]{scrartcl}

%%%%%%%%%%%%%%%%%%%%%%%%%%%%%% Formatierung %%%%%%%%%%%%%%%%%%%%%%%%%%%

%keine Einrückung nach leerzeile
\parindent0pt

% Für Kopf und Fußzeilen, siehe auch KOMA-Skript Doku
\usepackage[komastyle]{scrpage2}
\pagestyle{scrheadings}
\setheadsepline{0.5pt}[\color{black}]
\automark[section]{chapter}

%Zitate und Literaturverzeichnis
\usepackage[backend=bibtex,natbib=true,sorting=nyt,style=numeric-comp]{biblatex}
\usepackage[babel,german=quotes]{csquotes}
\bibliography{literatur}

%Zur vernünftigen Dekodierung
\usepackage[T1]{fontenc} %
\usepackage[utf8]{inputenc} %utfx8
\usepackage[ngerman]{babel} %

%Interaktives Dokument
\usepackage[pdfpagelabels=true]{hyperref}%

%Für wissenschaftliches Zitieren
%\usepackage{natbib}

%Schriftarten
%\usepackage{lmodern} %

%Formatierung für Kof- und Fußzeile. Hier gilt entweder ... oder ...!!

%Für eigenen Zeilenabstand
\usepackage{setspace} %

%Für die Seitenformatierung
\usepackage{lscape} %
\usepackage{multicol} %
\usepackage{wallpaper} %

%Styling Inhaltsverzeichnis
\usepackage{tocloft} %

% Zur Formatierung für Kopf und Fußzeilen. Im Allgemeinen ist scrpage2 besser als fancyhdr
\usepackage{scrpage2}
\pagestyle{scrheadings}
\setheadsepline{0.5pt}[\color{black}]

%Einstellungen für Figuren- und Tabellenbeschriftungen
\setkomafont{captionlabel}{\sffamily\bfseries}
\setcapindent{0em} 


%%%%%%%%%%%%%%%%%%%%%%%%%%%%%% Mathematisches %%%%%%%%%%%%%%%%%%%%%%%%%%%

%Pakete für Mathesymbole
\usepackage{latexsym,exscale,stmaryrd} %
\usepackage{amssymb, amsfonts, amstext} %
\usepackage{amsmath, mathtools, amsthm} %

%align nummerierung
\numberwithin{equation}{subsection}

% Weitere Symbole
\usepackage[nointegrals]{wasysym} %
\usepackage{eurosym} %
\usepackage{textcomp} %

%\usepackage{ucs} %

%Für vernünftige Einheiten 
\usepackage[separate-uncertainty, exponent-product = \cdot]{siunitx}
%\usepackage[thinspace,thinqspace,amssymb]{SIunits} %
\usepackage{icomma} %
\usepackage{nicefrac}%

%SI-Einheiten
\usepackage{siunitx}

%%%%%%%%%%%%%%%%%%%%%%%%%%%%%% Grafiken & Tabellen %%%%%%%%%%%%%%%%%%%%%%%%%%%
% Text umfließt Graphiken und Tabellen
% Beispiel:
% \begin{wrapfigure}[Zeilenanzahl]{"l" oder "r"}{breite}
%   \centering
%   \includegraphics[width=...]{grafik}
%   \caption{Beschriftung} 
%   \label{fig:grafik}
% \end{wrapfigure}
% Mehrere Abbildungen nebeneinander
% Beispiel:
% \begin{figure}[htb]
%   \centering
%   \subfigure[Beschriftung 1\label{fig:label1}]
%   {\includegraphics[width=0.49\textwidth]{grafik1}}
%   \hfill
%   \subfigure[Beschriftung 2\label{fig:label2}]
%   {\includegraphics[width=0.49\textwidth]{grafik2}}
%   \caption{Beschriftung allgemein}
%   \label{fig:label-gesamt}
% \end{figure}

%Subfigure nur mit PDF statt Bildern einfügen:
\usepackage{adjustbox}
%\begin{figure}[h]
%  \centering
%  \subfigure[Caption1\label{fig:bild1}]
%  {\begin{adjustbox}{width=0.44\linewidth}\input{bild1}\end{adjustbox}}
%  \hfill
%  \subfigure[Caption2\label{bild2}]
%  {\begin{adjustbox}{width=0.44\linewidth}\input{bild2}\end{adjustbox}}
%  \hfill
%  \subfigure[Caption3\label{bild3}]
%  {\begin{adjustbox}{width=0.44\linewidth}\input{bild3}\end{adjustbox}}
%  \caption{Gesamtcaption}
%  \label{fig:gesamtlabel}
%\end{figure}

%\usepackage{subfigure}




% Caption neben Abbildung
% Beispiel:
% \sidecaptionvpos{figure}{"c" oder "t" oder "b"}
% \begin{SCfigure}[rel. Breite (normalerweise = 1)][hbt]
%   \centering
%   \includegraphics[width=0.5\textwidth]{grafik.png}
%   \caption{Beschreibung}
%   \label{fig:}
% \end{SCfigure}

%Einstellungen für Figuren- und Tabellenbeschriftungen
\setkomafont{captionlabel}{\sffamily\bfseries}
\setcapindent{0em}

%Fuer mehr Platz in den Tabellen
\usepackage{cellspace} %mehr Platz in Tabellen
\addtolength\cellspacetoplimit{3pt}
\newcommand\myhline[1][2pt]{\\[#1]\hline}

%Zum Einbinden von GRafiken
\usepackage{graphicx}% [pdflatex]
\usepackage{xcolor}%

%Für textumflossene Grafiken
\usepackage{wrapfig} %

%Für subfigure
\usepackage{caption}
\usepackage{subcaption}

% Caption neben Abbildung
\usepackage{sidecap}

%Für URLs
\usepackage{url}%

%Zum Einbinden von Quelltext
\usepackage{listings-ext} %

%Für chemische Formeln
\usepackage{chemfig} %
%Für chemische Formeln (von www.dante.de)
%% Anpassung an LaTeX(2e) von Bernd Raichle
\makeatletter
\DeclareRobustCommand{\chemical}[1]{%
  {\(\m@th
   \edef\resetfontdimens{\noexpand\)%
       \fontdimen16\textfont2=\the\fontdimen16\textfont2
       \fontdimen17\textfont2=\the\fontdimen17\textfont2\relax}%
   \fontdimen16\textfont2=2.7pt \fontdimen17\textfont2=2.7pt
   \mathrm{#1}%
   \resetfontdimens}}
\makeatother
%erzwinge Fussnote auf selber Seite
\interfootnotelinepenalty=1000

%Zusätzliche Boxen
\usepackage{fancybox}

%\usepackage{framed}
%\usepackage{mathmode}
%\usepackage{empheq}

%Für variable Referenzen
\usepackage{varioref}

%Für Tabellen mit fester Gesamtbreite und variabler Spaltenbreite (im Gegensatz zu tabular)
\usepackage{tabularx}
%\newcommand{\ltab}{\raggedright\arraybackslash} % Tabellenabschnitt linksbündig
%\newcommand{\ctab}{\centering\arraybackslash} % Tabellenabschnitt zentriert
%\newcommand{\rtab}{\raggedleft\arraybackslash} % Tabellenabschnitt rechtsbündig


%Für Gleitobjekte
\usepackage{float} %Für H-Option

\usepackage{multirow} % Zellen von Tabellen zusammenfassen
\usepackage{booktabs} % verschoenert Tabellen
\usepackage{fixltx2e} % Repariert einige Dinge in Bezug auf das setzen von Gleitobjekten http://ctan.org/pkg/fixltx2e
\usepackage{stfloats} % Bei Gleitobjekten (figure,table,...) die ueber zwei Spalten gesetzt werden (Umgebung figure*), funktioniert [tb] http://ctan.org/pkg/stfloats
\usepackage{rotating} % Wird für Text und Grafiken benötigt, die um einen Winkel gedreht werden sollen



%%%%%%%%%%%%%%%%%%%%%%%%%%%%%% Kommandodefinitionen %%%%%%%%%%%%%%%%%%%%%%%%%%%

%Zur Korrektur und Kommentierung
\newcommand{\comment}[1]{\marginpar{\tiny{\textcolor{red}{#1}}}} % ermoeglicht kleine Kommentare am Seitenrand: \comment{Fehler?}
\newcommand{\Comment}[1]{\textcolor{red}{#1}}

%Zur Formatierung in der Matheumgebung
\renewcommand{\t}{\ensuremath{\rm\tiny}} % Tiefgestellter Text in der Matheumgebung wird schoener mit: $\Phi_{\t{Text}}$
\renewcommand{\d}{\ensuremath{\mathrm{d}}} % Die totale Ableitung ist stets aufrecht zu setzen: \d
\newcommand{\diff}[3][]{\ensuremath{\frac{\d^{#1}#2}{\d#3^{#1}}}} % einfache Ableitung nach x: $\ddx{\Phi}$
\newcommand{\pdiff}[3][]{\ensuremath{\frac{\partial^{#1}#2}{\partial#3^{#1}}}} % wie gesprochen, eine partielle Ableitung: \del
\newcommand{\aeqiv}{\ensuremath{\qquad \Longleftrightarrow \qquad}} % Eine Aequivalenz
\newcommand{\folgt}{\ensuremath{\qquad \Longrightarrow \qquad}} % Ein Folgepfeil mit Abstaenden
\newcommand{\corresponds}{\ensuremath{\mathrel{\widehat{=}}}} % Befehl für "Entspricht"-Zeichen
\newcommand{\mi}[1]{\ensuremath{\mathit{#1}}} % italics für griechische Buchstaben in Matheumgebung

%Um nicht so viel schreiben zu müssen...
\newcommand{\bs}[1]{\boldsymbol{#1}}
\newcommand{\ol}[1]{\overline{#1}}
\newcommand{\wtilde}[1]{\widetilde{#1}}
\newcommand{\mrm}[1]{\mathrm{#1}}
\newcommand{\mbf}[1]{\mathbf{#1}}
\newcommand{\mbb}[1]{\mathbb{#1}}
\newcommand{\mcal}[1]{\mathcal{#1}}
\newcommand{\mfrak}[1]{\mathfrak{#1}}

%Abkürzungen
\newcommand{\zB}{z.\,B.\ }
\newcommand{\bzw}{b.\,z.\, w.\ }
\newcommand{\Dh}{d.\,h.\ }
\newcommand{\Gl}{Gl.\ }
\newcommand{\Abb}{Abb.\ }
\newcommand{\Tab}{Tab.\ }

%Farbige Box um eine Formel
%Anwendung:
%\eqbox{
%  \begin{equation}
%    ...
%  \end{equation}
%}
\newcommand{\eqbox}[1]{
  \colorbox{gray!30}{\parbox{\linewidth}{#1}} 
}

%Im Text
\newcommand{\engl}[1]{engl. \textit{#1}}
\newcommand{\zitat}[1]{\footnote{#1}}
\newcommand{\person}[1]{\textsc{#1}}


%Matheoperatoren
\DeclareMathOperator{\tr}{tr}
\DeclareMathOperator{\sgn}{sgn}
\DeclareMathOperator{\diag}{diag}
\DeclareMathOperator{\const}{const}
\DeclareMathOperator{\grad}{grad}
\DeclareMathOperator{\rot}{rot}
\DeclareMathOperator{\divz}{div}


%%%%%%%%%%%%%%%%%%%%%%%%% Quellcode - Formatierung %%%%%%%%%%%%%%%%%%%%%%%%%%%%%%%%%%%%%%

%Um auch Umlaute in den Kommentaren auswerten zu können
\lstset{
literate = {Ö}{{\"O}}1 {Ä}{{\"A}}1 {Ü}{{\"U}}1 {ß}{{\ss}}2 {ü}{{\"u}}1
           {ä}{{\"a}}1 {ö}{{\"o}}1
}

%Formatierung des Quellcode
\lstset{
language=C++,
basicstyle=\footnotesize\ttfamily,
keywordstyle=\bfseries\color{blue},
stringstyle=\color{red},
commentstyle=\itshape\color{green!60!black},
emphstyle = \bfseries\color{red!80!green!60!blue}
%identifierstyle=,
}

%Zum Hervorheben bestimmter Begriffe (z.B. eigene Klassen, etc.)
%\lstset{
%emph = {vector, iterator, std, ostream, istream , ofstream, ifstream, fstream, cmath}
%}

%Nummerirung
\lstset{
numbers=left,
numberstyle=\tiny,
stepnumber=2,
numbersep=5pt,
frame=single,
breaklines=true
framesep=5pt,
numbersep=8pt,
breakindent=3ex
}

%Einbunden über
%\lstinputlisting[caption={blablabla}, language=C++]{name.cpp}

\begin{document}
%Autor, etc.
\newcommand{\versuch}{Version 2.9}
\newcommand{\titel}{2.9}
\newcommand{\praktikantA}{Eric Bertok}
\newcommand{\praktikantB}{Kevin Lüdemann}
\newcommand{\betreuer}{Phillip Bastian}
\newcommand{\emailA}{
      \href{mailto:eric.bertok@stud.uni-goettingen.de }
           {eric.bertok@stud.uni-goettingen.de} }
\newcommand{\emailB}{
      \href{mailto:kevin.luedemann@stud.uni-goettingen.de}
           {kevin.luedemann@stud.uni-goettingen.de} }
\newcommand{\gruppe}{B001}
\newcommand{\durchfuehrungsdatum}{11.07.2015}
\newcommand{\abgabedatum}{<++>}

%Metainformationen
\hypersetup{
      pdfauthor = {\praktikantB~ },
      pdftitle  = {Bedienungsanleitung},
      pdfsubject = {\titel}
}

\begin{titlepage}
\centering
\textsc{\Large Sonnenaufgangsweckerlampe\\[1.5ex] ++Firma einfügen++}

\vspace*{2.5cm}

\rule{\textwidth}{1pt}\\[0.5cm]
{\huge \bfseries
  \versuch\\[1.5ex]
}
\rule{\textwidth}{1pt}

\vspace*{3cm}

\begin{Large}
\begin{tabular}{ll}
Autor &  \praktikantB\\
Email:	& \emailB\\
Firma: & \gruppe\\
Version: & \titel\\
Ausgegeben: & \durchfuehrungsdatum\\
\end{tabular}
\end{Large}


\end{titlepage}

\cleardoublepage
\tableofcontents
\thispagestyle{empty}
\cleardoublepage

\setcounter{footnote}{0}
\setcounter{page}{1}

\pagenumbering{arabic}

\newpage

\section{Sicherheitshinweise}
\label{sec:sicherheitshinweise}
Dieses Produkt arbeitet mit Feinelektronik und es ist deshalb auf ESD zu achten.
Eine ungwollte Entladung kann wichtige Bauteile beschädigen und zum ausfallen der Funktionsfähigkeit führen.
Einige Bauteile, wie z.B. Kondensatoren oder Widerstände, könne ohne weiteres ersetzt werden, aber mach andere nicht.
Inbesondere die IC kann nicht ohne weiteres getauscht werden, da zu dem auch das Hauptprogramm neu aufgebrannt werden muss.
Um dies zu vermeiden, ist darauf zu achten keine offenen oder blanken Drähte oder Lötstelle auf der Platine zu berühren ohne sich vorher zu entladen.\\
Desweiteren können einige Bauteile im Betrieb sehr heiß werden.
Inbesondere die Stromversorgung ist im Betrieb der LEDs recht hohen Strömen ausgesetzt und dies führt zum rapiden erhitzen des Spannungsreglers.
Um sich vor Überhitzung zu schützen schaltet dieser ab einer zu hohen Temperatur ab.
Dies führt zum kurzzeitigen Ausfall der Funktion.
Um dies zu verhindern, wird angeraten das Gerät nicht in zu warmen Umgebungen zu betreiben.


\section{Kurzanleitung}
\label{sec:kurzanleitung}
Dies ist ein Wecker und zu dem auch noch eine Lampe.
Beide Funktionen können parallel verwendet werden.
Zudem wird die aktuelle Temperatur auf dem Display angezeigt.
Die interne Uhr hat eine Fehler in Sekunden von maximal 10 Sekunden pro Monat.

\subsection{Spannungsversorgung}
Er wird über eine externe Schnittstelle mit einer Spannung versorgt, die größer als 6.5\si{\volt} sein muss, um volle Funktionsfähigkeit zu gewärleisten.
Die Hardware ist für den Betrieb an einer 9\si{\volt} Blockbatterie ausgelegt.
Es kann aber über einen Adapter jede Spannungsquelle mit einem 3.5\si{\milli\meter} Stecker verwendet werden.

\subsection{Bedienung}
Die Bedienung läuft über die drei Taster an der Oberseite.
Der linke ist der "`+"' und "`Weckeraus"', der mittlere der "`Positions"' und "`Licht"' und der rechte ist der "`Menü"' Button.
%In späteren versionen ist rechts auch für die Display aus fkt
Das Display zeigt in jedem Fall die momentane Aktion an und ist in Deutsch geschrieben.

\subsection{erster Start}
Bei jedem Start des Wecker muss man die Weckzeiten und momentane Uhrzeit neu einstellen.
Zu Beginn erscheint ein Begrüßungsfenster und die Versionsnummer, aktuell Version: \titel.
Nach diesem wird der Benutzer aufgefordert die Werktags- und Wochenendsweckzeit einzustellen.
Ist dies geschehen, wird die Uhrzeit eingestellt.
Die Sekunden, weden dierekt nach der Eingabe der Werktages auf null gesetz und der Wecker nimmt seine Funktion auf.


\newpage
\section{Bedienungsanleitung}
\label{sec:Bedienungsanleitung}

\subsection{Beschreibung der Hardware}
Die Hardware besteht aus dem Gehäuse mit den LEDs und der Hauptpaltine.
Auf dem Deckel ist ein Display zur Zeitanzeige und Navigation durch das Menü vorhanden, sowie 3 Taster zu Steuerung.
Ebenfalls ist ein kleiner Lautsprecher zur ausgabe eines Wecktons vorhanden.\\
Die LEDs sind in Blöcken angeordnet.
Die oberen 4 LEDs, die unteren mittleren 3 roten LEDs und die beiden äußeren LEDs in der unteren Reihe gehören jeils zu einem Block.\\
Der linke Taster ist der "`+Taster"' zum erhöhen von Zahlenwerten und er "`Weckertaster"' zum auschalten des Weckers.
Der mittlere Taster ist der "`Postionstaster"' zur Navigation durch die Menüpunkte der verschiedenen Menüs und zu dem noch der "`Lichttaster"' zum ein- und ausschalten des Lichts.
Der rechte Taster ist der "`Menütaster"' mit diesem gelangt man aus und in das Hauptmenü.


\subsection{Inbetriebnahme}
\label{sec:inbetriebname}
Sobald eine Spannungsquelle an den Wecker angeschlossen ist und dieser über den Schieberegler eingeschaltet wird, erscheint der Willkommenschirm.
Dieser zeigt als Zusatzinformation die Versionsnummer an, momentan Version \titel.
Anschließend erscheint das Einstellungsmenü für den Werktagswecker.
Mit dem linken Taster erhöht man die Zeit und mit dem mittleren Spring man zur nächsten.
Nacheinander werden somit, die Stunden und Minuten des Werktagsweckers eingestellt.
Es ist zu beachten, dass der eingestelle Zeitpunkt, derjenige ist, an dem über den Lautsprecher ein heller Ton abgespielt wird.
Die Lichter beginnen bereits 20 Minuten füher an zu leuchten.\\
Anschließend wird nach dem gleichen verfahren der Wecker für das Wohenende eingestellt.
Wieder zuerst die Stunden und anschließend die Minuten.
Ist dies geschehen, wird der Wecker aktiviert und die Zeit kann eingestellt werden.\\
Bei der Einstellung der Zeit, wird genauso, wie bei den Weckzeiten vorgegangen.
Zuerst wird die Stunde und dann die Minuten eingestellt.
Anschließend wird der Werktag eingestellt und durch ein weiteren drücken auf dem "`Positionstaster"' die Sekunden auf null gestetzt.
Jetzt nimmt der Wecker seinen eigentlichen betrieb auf.\\
Ist der Wecker in aktiv und beginnt zu läuchten, so werden auf dem Hauptbildschirm in der oberen Zeile am Ende die Zeichen "`We"' eingeblendet.



\subsection{Wecker einstellen}
\label{sec:weckereinstellen}
Um den Wecker einzustellen, sei es nun den Werktags oder Wochenendswecker, muss man mit dem rechten Taster, dem "`Menütaster"' in das Menü gehen.
Die Einstellung der beiden Wecker sind Punkt 1 und 3 im Menü.
Die Einstellung, erfolgt genau, wie im Kapitel \ref{sec:inbetriebname} beschrieben.
Man navigiert zu dem gewünschten Einstellungspunkt und Drückt den linken, den "`+Taster"'.
Über diesen gelangt man in das Einstellungsmenü und wird aufgefordert die Weckstunde und durch einen Druck auf den "`Positionstaster"' die Weckminuten einzustellen.
Um Verwirrungen vorzubeugen, sind die Einstellungen der Weckzeiten durch "`WE"' und "`WO"' für Werktags und Wochenends gekennzeichnet.\\
Sind die richtigen Zeiten eingestellt, reicht ein Druck auf den "`Positionstaster"' und man gelangt zurück in das Hauptmenü.
Die aktuelle Weckzeit und die Lichtzeiten sind nun berechnet und der Wecker ist aktiviert.
Von hier aus gelangt man über ein weiteren Drücken der "`Menütaste"' zurück in den Hauptbildschirm.\\
Ist der Wecker einmal aktiv und die Lampen leuchten, bzw. der Lautsprecher spielt den hellen Ton ab, kann nicht in das Menü gegangen werden.
Erst, wenn der Wecker über die "`Weckertaste"' kann der Wecker ausgeschaltet werden und die Menüfunktion wieder aktiviert werden.
Ist der Wecker in einer der beiden frühen Lichtphasen beendet, so wird er nicht durch die letzte Phase gehen und den Ton abspielen.

\subsection{Wecker auschaltelten}
Um einer der Wecker oder beide auszuschalten, wird wieder über die "`Menütaste"' das Hauptmenü geöffnet und zum Punkt 2 bzw. 4 navigiert.
Hier hat man jetzt die Möglichkeit den Werktags und Wochenendswecker auszuschalten.
Eine $1$ in der zweiten Zeile bedeutet, der Wecker ist aktiv und eine $0$, dass der Wecker inaktiv ist.
Dies kann jederzeit geändert werden.
Jedoch ist zu beachten, dass wenn der Wecker im eigentlichen Weckintervall eingeschaltet wird, keine Weckfunktion besteht.
Dies liegt daran, dass standartmäßig der Wecker 20 Minuten vor der Weckzeit beginnt das Licht einzuschalten.

\subsection{Weckerverzögerung einstellen}
Wie bereits ober erwähnt schaltet der Wecker 20 Minuten vor der Weckzeit das Licht an.
Dies geschieht in zwei aufeinander folgenden Phasen.
Zuerst wird nur das halbe Licht eingeschaltet und nach ablauf der halben noch fehlenden Zeit, das ganze Licht.
Dies sorgt für ein langsames, nicht schlagartiges aufwachen.\\
Das Zeitintervall, in dem das Licht aktiv ist, kann über das Menü im Punkt 5 eingestellt werden.
Hierzu betätigt man den "`Menütaster"' und navigiert mit dem "`Positionstaster"' zum Menüpunkt der Weckverzögerung.
Hier kann jetzt durchdrücker der "`+Taste"' die Zahl erhöht werden.
Nach jedem Drücken wird die Zahl aktualiesiert und es kann die Menüposition gewechselt werden.\\
Die maximale Weckverzögerung ist 60 Minuten und die minimale 10 Minuten.
Nach dem umschalten, ist zu beachten, dass die Weckzeiten immernoch mit der alten Verzögerung berechnet wurden.
Es ist also nötig diese zu aktualisieren.
Dazu muss man, wie im Kapitel \ref{sec:weckereinstellen} beschrieben durchführen.
Hierbei ist es aber nicht notwendig die Weckzeit zu ändern, wenn nicht erwünnscht.
Durch Wechsel der Einstellposition mit dem "`Postionstaster"' ohne die Zahlen zu ändern, kann die Weckzeit behalten werden und nur die Verzögerung für das Licht geändert werden.
Dies muss allerdings für den Werktags- und den Wochenendswecker durchgeführt werden.

\subsection{Zeit einstellen}
Um die Zeit einzustellen, öffnet man das Hauptmenü mit der "`Menütaste"' und navigiert mit der "`Positionstaste"' zum 6 Punkt.
Die man gelangt nun über durch einen Druck auf die "`+Taste"' und das Einstelungsmenü für die Zeit.
Hier kann jetzt nach und nach die Stunde, die Minute und der Werktag mithilfe der "`+Taste"' eingestellt werden.
Um von einem Einstellungspunkt zum nächsten zu gelangen reicht ein Druck auf die "`Positionstaste"'.
Nach Einstellung des Werktages und letzten Druck auf die "`Postionstaste"' werden die Sekunden auf null gestzt und die Zeit läuft mit der Aktuellen weiter.
Man befindet sich nun wieder im Hauptmenü und kann dieses wieder mit einem Druck auf die "`Menütaste verlassen"'.


\subsection{Sekundenoffset einstellen}
Der Wecker hat einen gemessenen Sekundenfehler von etwa 2 Sekundne pro Woche.
Dies wird durch das zurücksetzten der Zeit von 2 Sekunden am ende jeder Woche behoben.
Dies geschieht jeden Samstag um exakt 0 Uhr.
Dieser Offset kann über das Menü angepasst werden.
Standartmaßig ist ein Offset von -2 engegeben.
Diese Zahl wird jeden Samstag auf die Sekundenzahl draufgerechnet.
Sprich in diesem Fall abgezogen.\\
Die Einstellung dieses Offset ist Menüpunkt 7.
Auch hier wird durch betätigen des "`+Tasters"' die Zahl um eins erhöht.
Nach jedem erhöhen, kann weiter durch das Hauptmenü navigiert werden, da dieses nicht erlassen wird.
Es ist möglich ein Sekundenoffset von $\pm5$ Sekunden einzustellen.


\subsection{Lampe verwenden}
Um die Lampe zu verwenden, wird auf dem Hauptbildschirm die "`Lampetaste"' gedrückt.
Die Lampen gehen sofort an und können auf die gleiche Weise ausgeschaltet werden.
Um zu unterscheiden, ob die Lampe angeschaltet sind, oder ob der Wecker aktiv ist, befindeen sich am Ende der ersten Zeile des Displays die Zeichen"`Li"'.
Das Licht kann nur aus dem Hauptbildschirm aus eingeschaltet und abgeschaltet werden.

\subsection{Lampenstärke ändern}
Es ist Möglich einzustellen, wieviele Lampen leuchten.
Die ist über den Menüpunkt 9 zu erreichen.
Dazu öffnet man das Hauptmenü mit der "`Menütaste"' und navigiert mit der "`Positionstaste"' zum Punkt 9 Lamepnstärke.
Hier kann jetzt die Zahl mithilfe der "`+Taste"' um eins erhöht werden und anschließend weiter im Menü navigiert werden, da dieses nicht verlassen wird.\\
Es besteht die Möglichkeit die Follgende Anzahl an Lichtern zu schalten.
Eine 1 bedeutet, dass nur die oberen 4 Lichter leuchten.
Eine 2 bedeutet, dass die oberen 4 und die unteren drei roten Licihter leuchten.
Eine 3 wiederum bedeutet, dass alle lichter Leuchten.

\subsection{Temperaturoffset einstellen}
Die angezeigte Temperatur in \si{\celsius} ist bereits geeicht und linear.
Das bedeutet, dass die angezeigte Temperatur mit einem Fehler von $\pm1\si{\celsius}$ genau ist.
Die angezeigte Temperatur ist auf die zentel Stelle gerundet, da keine höhere Genauigkeit erreicht werden kann.\\
Ist die Temperatur doch mal nicht ganz richtig, aufgrund von konstanen Stromschwankungen, oder fahlschen Verbindungen, so kann dies über die Addition eines konstanten Offsets korrigert werden.\\
Dieser kann über den Menüpunkt 8 eingestellt werden.
Hierzu wird wieder über die "`Menütaste"' das Hauptmenü geöffnet und mit der "`Positionstaste"' zum Menüpunkt Temperaturoffset navigiert.
Hier kann jetzt der Offset durch das Betätigen des "`+Tasters"' um eins erhöht werden.
Anschließend, kann wieder weiter durch das Hauptmenü navigiert werden, da dieses nicht verlassen wird.
Es ist Möglich einen Offset von bis zu $\pm10\si{celsius}$ einzustellen.


\section{Wartungsanleitung}
\label{sec:wartungsanleitung}
Zur Wartung des Gerätes ist, zu beachtet, dass dieser nicht Wasserdicht ist und durch den Kontakt mit Wasser beschädigt werden kann.
Es ist hin und wieder nötig die LEDs zu entstauben, damit die weiterhin volle Leuchtkraft haben.
Desweiteren ist darauf zu achten, dass keine Kabel blanke stelle besitzen, oder anfangen sich von der Platine oder den Bauteilen zu lösen.

\section{Garantiehinweise}
\label{sec:Garantiehinweise}
Garantie ist ausgeschlossen, allerdings wird das Produkt repariert, sollte es jemals Schaden nehmen.
Über die Kosten der Reparatur wird von Fall zu Fall neu verhandelt.
Es ist ebenfalls zu beachten, dass jeder augelieferte Wecker ein Einzelstück ist und Hangefertigt.

\newpage
%\nocite{*} %sorgt dafuer, dass alles ausgegeben wird
\printbibliography[heading=bibintoc]
\end{document}
